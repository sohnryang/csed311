\documentclass{scrartcl}
\usepackage[margin=3cm]{geometry}
\usepackage{amsmath}
\usepackage{amssymb}
\usepackage{amsthm}
\usepackage{blindtext}
\usepackage{datetime}
\usepackage{fontspec}
\usepackage{float}
\usepackage{graphicx}
\usepackage{kotex}
\usepackage[lighttt]{lmodern}
\usepackage{listings}
\usepackage{mathrsfs}
\usepackage{mathtools}
\usepackage{pgf,tikz,pgfplots}

\pgfplotsset{compat=1.15}
\usetikzlibrary{arrows}
\newtheorem{theorem}{Theorem}

\lstset{
  numbers=none, frame=single, showspaces=false,
  showstringspaces=false, showtabs=false, breaklines=true, showlines=true,
  breakatwhitespace=true, basicstyle=\ttfamily, keywordstyle=\bfseries, basewidth=0.5em
}

\setmainhangulfont{Noto Serif CJK KR}[
  UprightFont=* Light, BoldFont=* Bold,
  Script=Hangul, Language=Korean, AutoFakeSlant,
]
\setsanshangulfont{Noto Sans CJK KR}[
  UprightFont=* DemiLight, BoldFont=* Medium,
  Script=Hangul, Language=Korean
]
\setmathhangulfont{Noto Sans CJK KR}[
  SizeFeatures={
    {Size=-6,  Font=* Medium},
    {Size=6-9, Font=*},
    {Size=9-,  Font=* DemiLight},
  },
  Script=Hangul, Language=Korean
]
\title{CSED311: Lab 2 (due Mar. 26)}
\author{김태연(20220140), 손량(20220323)}
\date{Last compiled on: \today, \currenttime}

\newcommand{\un}[1]{\ensuremath{\ \mathrm{#1}}}

\begin{document}
\maketitle

\section{Introduction}
RISC-V architecture를 바탕으로 구성된 Single cycle CPU가 작동하는 방식을 알아보고,
각 Instruction type에 따라 거치는 Datapath에 유의하며 CPU를 구성하는 각 module을
연결하여 Single cycle CPU를 구현한다.

\section{Design}
이번에 구현한 RISC-V Single cycle CPU는 다음과 같은 Submodule로 나누어서 설계하였다.

\subsection{ALU}

\subsection{Register file}

\subsection{Data memory}

\subsection{Instruction memory}

\section{Implementation}

각 베릴로그 파일에 대한 설명은 다음과 같다.

\subsection{\texttt{cpu.v} -- 내부적인 Module을 연결하여 CPU 구성}

\subsection{\texttt{cpu\_def.v} -- CPU control unit을 위한 상수 정의}

\subsection{\texttt{opcodes.v} -- CPU instruction set에 대응되는 상수 정의}

\subsection{\texttt{control\_unit.v} -- Instruction으로부터 CPU control signal 생성}

\subsection{\texttt{register\_file.v} -- Register file 구현}

\subsection{\texttt{instruction\_memory.v} -- Instruction memory 구현}

\subsection{\texttt{data\_memory.v} -- Data memory 구현}

\subsection{\texttt{alu.v} -- ALU 동작 구현}

\subsection{\texttt{alu\_def.v} -- ALU 구성을 위한 상수 정의}

\subsection{\texttt{alu\_control\_unit.v} -- Instruction과 ALU opcode 대응}

\subsection{\texttt{immediate\_generator.v} -- Instruction type 별 immediate value 생성}

\subsection{\texttt{pc.v} -- Program Counter 동작 구현}

\subsection{\texttt{adder32bit.v} -- 내부적인 Adder 동작 구현}

\subsection{\texttt{mux32bit.v} -- 내부적인 Multiplexer 동작 구현}

\subsection{\texttt{top.v} -- Top module}

\section{Discussion}

\section{Conclusion}

\end{document}
% vim: textwidth=79
