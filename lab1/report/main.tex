\documentclass{scrartcl}
\usepackage[margin=3cm]{geometry}
\usepackage{amsmath}
\usepackage{amssymb}
\usepackage{amsthm}
\usepackage{blindtext}
\usepackage{datetime}
\usepackage{fontspec}
\usepackage{float}
\usepackage{graphicx}
\usepackage{kotex}
\usepackage[lighttt]{lmodern}
\usepackage{listings}
\usepackage{mathrsfs}
\usepackage{mathtools}
\usepackage{pgf,tikz,pgfplots}

\pgfplotsset{compat=1.15}
\usetikzlibrary{arrows}
\newtheorem{theorem}{Theorem}

\lstset{
  numbers=none, frame=single, showspaces=false,
  showstringspaces=false, showtabs=false, breaklines=true, showlines=true,
  breakatwhitespace=true, basicstyle=\ttfamily, keywordstyle=\bfseries, basewidth=0.5em
}

\setmainhangulfont{Noto Serif CJK KR}[
  UprightFont=* Light, BoldFont=* Bold,
  Script=Hangul, Language=Korean, AutoFakeSlant,
]
\setsanshangulfont{Noto Sans CJK KR}[
  UprightFont=* DemiLight, BoldFont=* Medium,
  Script=Hangul, Language=Korean
]
\setmathhangulfont{Noto Sans CJK KR}[
  SizeFeatures={
    {Size=-6,  Font=* Medium},
    {Size=6-9, Font=*},
    {Size=9-,  Font=* DemiLight},
  },
  Script=Hangul, Language=Korean
]
\title{CSED311: Lab 1 (due Mar. 3)}
\author{김태연(20220140), 손량(20220323)}
\date{Last compiled on: \today, \currenttime}

\newcommand{\un}[1]{\ensuremath{\ \mathrm{#1}}}

\begin{document}
\maketitle

\section{Introduction}
베릴로그 RTL 설계 방법을 익히기 위해 간단한 combinational circuit과 sequential circuit인 ALU와
vending machine을 구현해 본다.

\section{Design}
Vending machine은 크게 다음과 같은 모듈으로 나누어 설계하였다.

\subsection{Coin Storage}
이름에서 유추할 수 있듯, 사용자가 넣은 동전을 관리하는 모듈이다. 이 모듈은 다음 입력을 처리할 수 있도록 설계하였다.

\begin{itemize}
  \item 사용자가 집어넣은 동전에 대한 입력
  \item 사용자가 구입한 물품의 가격
  \item 동전 반환 버튼 입력, 시간 초과에 해당하는 입력
\end{itemize}

Coin storage의 경우 사용자 입력에 대해 바로 출력이 일어날 필요가 없으므로 Moore machine 형태로 설계하였다. 사용자의
입력만 갖고 다음 상태를 계산하기 위해, 상태를 저장하는 레지스터에 대해 다음 상태를 담는 레지스터를 하나 더 추가해 구현하였다. 이 모듈에서
사용하는 레지스터는 다음 정보를 저장한다.

\begin{itemize}
  \item 현재 저장하고 있는 동전의 총 합
  \item 반환할 동전의 조합
  \item 반환 버튼을 누르고 실제 동전 반환이 일어날 때까지 남은 시간
\end{itemize}

이러한 정보를 저장하는 것에서, coin storage가 가지는 상태는 다음과 같은 4-튜플으로 나타낼 수 있다.

\begin{align*}
  (C, R, T, S)
  \in \{0, 1\}^{32} \times \{0, 1\}^\texttt{kNumCoins} \times \{0, 1\}^2
  \times \{0, 1\}
\end{align*}

여기서 각 변수에 대한 설명은 다음과 같다.

\begin{itemize}
  \item \(C\) -- 현재 저장 중인 동전의 총 합
  \item \(R\) -- 반환할 동전의 조합
  \item \(T\) -- 반환 버튼을 누르고 실제 반환이 일어날 때까지 남은 시간
  \item \(S\) -- 반환 버튼을 누르고 기다리는 중인지의 여부
\end{itemize}

\subsection{Timekeeper}
일정 시간 이상 동전을 투입하지 않거나, 조작을 하지 않은 경우 현재 투입되었던 동전을 반환하고 초기 상태로 되돌아가야 한다.
만약 해당 기능을 메인 모듈의 내부에서 구현할 경우, 각 clock tick 마다 현재 동작을 확인해야만 하기 때문에 별도의 모듈로
구현하였다.
해당 모듈 내에서 사용하는 레지스터는 타이머의 남은 시간과 타이머의 갱신 여부를 저장한다. 타이머를 갱신하지 않아야 하는 경우는
다음과 같다.

\begin{itemize}
  \item 사용자 입력 이후 타이머가 0에 도달한 경우
  \item 사용자가 명시적으로 반환 버튼을 누른 경우
\end{itemize}

해당 경우 모두 사용자가 입력한 동전을 모두 반환하고, 타이머를 갱신하지 않는 상태로 되돌아간다. 이는 해당 모듈의
output register인 timeout에 신호를 출력함으로써 구현하였다. 해당 register는 Coin storage 모듈의 input으로
들어가 반환 동작을 하게 한다.

\subsection{계산을 위한 combinational logic}
동전 조합과 액수 사이의 변환에 대한 계산을 모두 sequential logic 내부에서 계산한다면 sequential logic의 코드가
복잡해진다. 이러한 상황을 피하기 위해 간단한 계산을 수행하기 위한 combinational logic은 별도의 모듈으로 구현하도록
설계하였다. 크게 다음과 같은 combinational logic 모듈을 설계하여 사용하였다.

\begin{itemize}
  \item 현재 자판기에 저장된 액수로 구매할 수 있는 물품 표시
  \item 사용자의 물품 선택을 입력받아 사용되는 액수를 계산
  \item 동전 조합을 입력받아 총 합을 계산
  \item 주어진 액수를 반환할 수 있는 동전의 조합을 계산
\end{itemize}

위에서 나열한 모듈 중 마지막 모듈의 경우에는 출력 조건 상, 각 동전 종류는 한 사이클에 최대 한 개만 반환할 수 있으므로 필요하다면
sequential logic에서 여러 번 입력 값을 바꾸어 가면서 출력 값을 이용하는 것을 가정하였다.

\section{Implementation}
각 베릴로그 파일에 대한 설명은 다음과 같다.

\subsection{\texttt{coin\_storage.v} -- Coin storage 모듈 구현}

\subsection{\texttt{timekeeper.v} -- Timekeeper 모듈 구현}

\subsection{\texttt{item\_display.v} -- 구매할 수 있는 물품 표시}

\subsection{\texttt{item\_selector.v} -- 사용자의 물품 선택 처리}

\subsection{\texttt{coin\_value\_calculator.v} -- 동전 조합에 따른 액수 계산}

\subsection{\texttt{return\_value\_calculator.v} -- 반환할 수 있는 동전 조합 계산}

\section{Discussion}
\texttt{:ta\_ggang:} % TODO: write discussion

\section{Conclusion}

\end{document}
% vim: textwidth=79
