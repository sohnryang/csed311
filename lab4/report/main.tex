\documentclass{scrartcl}
\usepackage[margin=3cm]{geometry}
\usepackage{amsmath}
\usepackage{amssymb}
\usepackage{amsthm}
\usepackage{blindtext}
\usepackage{datetime}
\usepackage{fontspec}
\usepackage{float}
\usepackage{graphicx}
\usepackage{kotex}
\usepackage[lighttt]{lmodern}
\usepackage{listings}
\usepackage{mathrsfs}
\usepackage{mathtools}
\usepackage{pgf,tikz,pgfplots}
\usepackage{pdfpages}
\usepackage{tabularx}

\pgfplotsset{compat=1.15}
\usetikzlibrary{arrows}
\newtheorem{theorem}{Theorem}

\lstset{
  numbers=none, frame=single, showspaces=false,
  showstringspaces=false, showtabs=false, breaklines=true, showlines=true,
  breakatwhitespace=true, basicstyle=\ttfamily, keywordstyle=\bfseries, basewidth=0.5em
}

\setmainhangulfont{Noto Serif CJK KR}[
  UprightFont=* Light, BoldFont=* Bold,
  Script=Hangul, Language=Korean, AutoFakeSlant,
]
\setsanshangulfont{Noto Sans CJK KR}[
  UprightFont=* DemiLight, BoldFont=* Medium,
  Script=Hangul, Language=Korean
]
\setmathhangulfont{Noto Sans CJK KR}[
  SizeFeatures={
    {Size=-6,  Font=* Medium},
    {Size=6-9, Font=*},
    {Size=9-,  Font=* DemiLight},
  },
  Script=Hangul, Language=Korean
]
\title{CSED311: Lab 4 (due Apr. 30)}
\author{김태연(20220140), 손량(20220323)}
\date{Last compiled on: \today, \currenttime}

\newcommand{\un}[1]{\ensuremath{\ \mathrm{#1}}}

\begin{document}
\maketitle

\section{Introduction}
RISC-V architecture를 바당으로 Control flow instructiond을 처리하지 않는 Pipelined CPU를 구현한다.

\section{Design}
이번에 구현한 RISC-V Multicycle CPU는 다음과 같은 주요 기능을 가진 Submodule로 나누었으며, 생성된 값들을 선택하기 위해
추가적으로 Multiplexer를 구현하여 사용했다. 기존 Single-cycle CPU에 추가적으로 Forwarding unit과 Forwarding path를
설계하여 Pipelining을 지원하였다.

\subsection{Program Counter}

\subsection{Instruction Memory and Data Memory}

\subsection{Register File}

\subsection{Inter-Stage Registers}
\subsubsection{IF--ID}
\subsubsection{ID--EX}
\subsubsection{EX--MEM}
\subsubsection{MEM--WB}

\subsection{ALU}

\subsection{Immediate Generator}

\subsection{Control Unit}

\subsection{Hazard Detection Unit}

\section{Hazard Detection and Data Forwarding}

\section{Implementation}
각 베릴로그 파일에 대한 세부 설명은 다음과 같다.

\subsection{\texttt{top.v} -- Top module}
\texttt{cpu.v}에서 구현한 \texttt{cpu} 모듈을 사용하여 Pipelined CPU를 구현하였다.

\subsection{\texttt{cpu.v} -- 내부 Module을 연결하여 Pipelined CPU 구성}
\texttt{cpu.v}는 다음과 같은 submodule을 연결하여 구성하였다.
\begin{itemize}
\end{itemize}

\section{Discussion}

\section{Conclusion}

\end{document}
% vim: textwidth=79
